\section{Study 1 - Hackerbot}
\subsection{Description}

\paragraph{}Hackerbot is a program that is used as part of Hacktivity and SecGen to play the role of an attacker or an examiner. It's code is included as part of the SecGen repository on GitHub. It works by interacting with students over an \acrfull{irc} server hosted in a virtual machine alongside hackerbot. They can then access hackerbot through an \acrshort{irc} client such as Pidgin through one of more \acrshort{VM}s that are part of an educational cybersecurity scenario environment such as a \acrshort{ctf}, cyberrange, or lab exercise. When prompted as part of a challenge is can then execute commands inside the \acrfull{VM} it is installed within to attack other \acrshort{VM}s in the training environment or to perform actions remotely. It is pre-programmed by SecGen during the creation and configuration of it's host \acrlong{VM} with a scenario file including all the challenges, prompt, and commands it will need for that specific exercise.

\paragraph{}Currently hackerbot are quite limited. It responds to specific commands, and any other input that does not fall within those is passed to a simple chatbot called \acrfull{alice}. \acrshort{alice} is an older generation technology that uses pre-determined responses when given a question. It does not use any form of \acrlong{dl} or \acrfull{ann}. The primary aim of this study is to make hackerbot more realistic. One of the ways of doing this is to get hackerbot to respond in a more human like manner to inputs outside of it's hard-coded commands by using \acrshort{llm} instead of \acrshort{alice}. Using more advanced chatbots may also allow for replacing of commands that need to be strictly typed and spelled to a more natural interface - though this would require the use of more advanced techniques such as agentic AI. It could also allow hackerbot to take on the role of specific characters that are 

\subsection{Methology}
\paragraph{}The new version of Hackerbot will connect to an Ollama backend and submit requests through that REST API.